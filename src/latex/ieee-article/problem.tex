\section{Problem Definition}
Here you are expected to define what is the subject of the study.

% Shows how to insert your algorithm and how to refer to it via `ref'
Algorithm~\ref{algo:factorial} illustrates the recursive algorithm for
the computation of the factorial.

% displays the algorithm for computing the factorial recursively:
\begin{algorithm}[H]	% uses the float package to control placement
	\caption{Factorial(N)}	% a brief description or the function name
	\begin{algorithmic}
	% here is where you use commands from the algorithmic package to
	% write your algorithm
		\IF {$N == 1$}
			\RETURN $1$	\COMMENT{optional: inline comment}
		\ELSE
		\STATE $p \gets$ Factorial$(N - 1)$
		\STATE $r \gets p * N$
		\ENDIF
		\RETURN r
	\end{algorithmic}
	\label{algo:factorial}	% defines a label to refer to this
\end{algorithm}
